\documentclass[9pt,a4paper]{article}
\usepackage[utf8]{inputenc}       
\usepackage[english,russian]{babel}
\usepackage{PTSerif}                
\usepackage[pdftex]{graphicx}        
\usepackage{layout}                   
\usepackage{fancyhdr}                  
\usepackage{fullpage}                   
\usepackage{array}                       
\usepackage{longtable}                    
\usepackage{listings}
\usepackage{footnote}                       
                                             
\setlength\voffset{-1in}                      
\setlength\hoffset{-1in}                       
\setlength\topmargin{1cm}                       
\setlength\oddsidemargin{2cm}                    
\setlength\textheight{25.7cm}                     
\setlength\textwidth{17.001cm}                     
\setlength{\topskip}{1.3cm}                         
\setlength\headheight{0cm}                           
\setlength\headsep{0cm}                               
                                                       
\pagestyle{fancyplain}                                  
\fancyhf{}                                               
\cfoot{\small\em \textcopyright \hspace{0.1em} ARCСN 2013}
\rfoot{\small \thepage}

\renewcommand{\labelitemii}{$\circ$}
                                      
\title{HCProbe Manual}
\author{Alexander Vershilov, Kirill Zaborsky}

\begin{document}
\maketitle
\begin{figure}[!h]
   \centering 
   \includegraphics[width=0.3\columnwidth]{images/testcfg2.png}
\end{figure}                                                        

\tableofcontents

\pagebreak

\section{Introduction}

\textbf{HCProbe} is a tool for testing OpenFlow controllers which
includes a library in Haskell providing means to work with OpenFlow
protocol. Also it includes a reference implementation of an OpenFlow
software switch and a domain-specific language for constructing new
custom switches.

\subsection{List of terms}

\begin{description}
  \item[OpenFlow] is a communications protocol that gives access to
    the forwarding plane of a network switch or router over the
    network.
  \item[EDSL] embedded domain-specific language
\end{description}

\pagebreak

\section{Subsystems description}

Project HCProbe includes the following subsystems:

\begin{description}
    \item[OpenFlow library] -- OpenFlow protocol implementation
    \item[OpenFlow/Ethernet] -- generation of Ethernet frames
      containing higher level protocols
    \item[HCProbe] -- reference OpenFlow switch implementation
    \item[HCProbe/EDSL] -- domain-specific language for constructing
      OpenFlow software switches
    \item[tests] -- packet generation and parsing tests
    \item[примеры] -- OpenFlow test examples which use switch
      construction EDSL mentioned above
\end{description}


\subsection{OpenFlow library}

OpenFlow library provides OpenFlow protocol implementation in Haskell
based on the <<OpenFlow Switch Specification, Version 1.0.0>>
\footnote{http://www.openflow.org/documents/openflow-spec-v1.0.0.pdf} standard.
Main library submodules are:

\begin{description}
    \item[Network.Openflow.Types] -- OpenFlow protocol types and data
      structures
        \begin{itemize}
            \item \textbf{Binary.Read} instances for parsing data
              structures
            \item \textbf{Enum} instances for enumerations
        \end{itemize}
    \item[Network.Openflow.Misc] -- miscellaneous helper functions
        \begin{itemize}
            \item CRC calculation
            \item serialization functions for some types
            \item functions dealing with IP amd MAC addresses
        \end{itemize}
    \item[Network.Openflow.Messages] -- OpenFlow message
      serialization and parsing functions.
\end{description}

\emph{Current implementation is not complete as not 100\% of all
  structures and messages from specification are implemented, but the library
  could be relatively easily extended by analogy with the already
  existing code.}

\subsection{OpenFlow/Ethernet}

This subsystem provides an interface for creating packets of different
network protocols (Ethernet, IPv4, ARP, TCP). With it one could
create custom test packets for incapsulation into PacketIn OpenFlow
messages.

Main modules:

\begin{description}
  \item[Network.Openflow.Ethernet] -- module reexporting all
    submodules (so you can import only this module)
  \item[Network.Openflow.ARP]      -- ARP packets
  \item[Network.Openflow.Frame]    -- network frames (Ethernet)
  \item[Network.Openflow.IPv4]     -- IPv4 packets
  \item[Network.Openflow.TCP]      -- TCP packets
  \item[Network.Openflow.Types]    -- internal data types
\end{description}

hcprobe is a program implementing a number of software switches
which could be used as a reference implementation of an OpenFlow
switch.

\texttt{hcprobe} program creates some specified number of virtual
switches which connect to a specified controller and after that they
begin to send PacketIn messages to that controller. These messages
contain Ethernet frames with a header including MAC addresses
corresponding to some of virtual switch ports. Also these Ethernet
frames contain an IP packet with a TCP segment in it.

While running the program outputs some statistics to the console. The data
shown includes the number of sent messages, received messages, mean
roundtrip time for one message and also the number of lost messages and
closed connections with controller.

\subsection{EDSL}

We developed HCProbe domain-specific language for creating OpenFlow
switches and different combinations of them, writing programs for the
switches and also for registration and collection of statistics on
packet exchange between OpenFlow controller and switches.

\subsubsection{Program structure}

Typical program created using HCProbe domain-specific language
includes at least the following 2 parts:

\begin{itemize}
  \item Switch creation and configuration
  \item Running the specified program for the switches
\end{itemize}

In most cases 3rd step is also required: the main program thread
waits for switch programs completion (for all or some of them
depending on current task) and calculates summarized statistics.

\subsubsection{Switch creation}

Switch could be created using either \lstinline!switch! or
\lstinline!switchOn! command.  In the first case a switch with default
settings is used. In the second case some previously created
switch could be supplied as a parameter. Thus using command
\lstinline!switchOn! one could create a copy of some already existing
switch.

In order to ensure that MAC address ranges do not overlap between
different switches environment \lstinline!config! should be used in
which switches must be created:

\begin{lstlisting}
    sw <- config $ switch <switchIP> $ do ..
\end{lstlisting}%$

\subsubsection{Configuring switch settings}

Switch configuration is done using environment \lstinline!features! in
which different switch parameters could be specified.

\begin{lstlisting}
    switch <switchIP> $ do
      features $ do
        ..
\end{lstlisting}

These parameters include ports added with \lstinline!addPort! command.

Command \lstinline!addMACs! adds a list of MAC addresses to
the switch. And those addresses are equally divided between different
ports of the switch they are assigned to.

\emph{As it was said above to prevent overlapping between MAC
  addresses of different switches environment \lstinline!config! is
  used. So if during the call to \lstinline!addMACs! some
  overlapping with the prevously configured switches will be detected then
  instead of these overlapping addresses that switch will be assigned
  other higher an not yet reserved MAC addresses}

\lstinline!clearMACs! clears any MAC addresses assigned to the switch.


\subsubsection{Switch run}

Switch could be run in 2 ways:

\begin{itemize}
  \item if the standard switch logic is enough for the task it should be
    run using command \lstinline!runSwitch!;
  \item more frequently one will need some custom switch behaviour which
    could be set up using command \lstinline!withSwitch!.
\end{itemize}


\subsubsection{Program run}

After the switch was started using \lstinline!withSwitch! it begins to
execute some program that was specified. Such program is just an
ordinary block of Haskell code which could work with switch using
special commands for that. These commands include:

\begin{description}
  \item[hangOn] -- wait forever, may be useful in some testing scenarios.

  \item[waitForType] -- waiting for some particular type of message
    from controller: program execution will be suspended till a
    message of the specified type arrives. Rreturns the received message.

  \item[waitForBID] -- waiting for a message containing the specified
    \lstinline!buffer id!.  Returns the received message.

\end{description}


For message generation it is desirable that transaction numbers and
buffer ids used in a switch program do not overlap with each
other. To solve this issue commands \lstinline!nextBID! and
\lstinline!nextTID! should be used to get proper ids.

For message sending the following commands could be used:

\begin{description}

  \item[send] -- send arbitrary OpenFlow message to controller

  \item[sendOFPacketIn] -- send OpenFlow message with type
    \lstinline!OFPT_PACKET_IN!

  \item[sendARPGreeting] -- used for sending message
    \lstinline!OFPT_PACKET_IN!  containing ARP reply inside (could be
    used when one needs to inform controller about all MAC
    addresses available to switch before actuall testing starts)

  \item[statsSend/statsSendOFPacketIn] -- variants of \lstinline!send!
    and \lstinline!sendOFPacketIn! including statistics collection

\end{description}

\subsubsection{Statistics collection}

Current hcprobe implementation supports collection of the following
statistical parameters:

\begin{itemize}
  \item number of sent messages
  \item number of messages for which reply from the controller was
    received
  \item number of messages with no reply from the controller (messages are
    assumed lost when the reply waiting queue overflows)
  \item round-trip times from message sending to receiving reply for it
\end{itemize}

Using statistics collection requires the following steps:

\begin{itemize}
  \item \lstinline!initPacketStats! execution to create
    \lstinline!StatsEntity! in which statistics data will be stored
  \item statistics handler regsitration using
    \lstinline!setSilentStatsHandler!  or \lstinline!setStatsHandler stEnt ...!.
    The second call is used e.g. to execute some action on recieving a reply
    for a message. Such action could include
    outputting round-trip time to the console or a file
  \item sending message to controller using commands
    \lstinline!statsSend! or \lstinline!statsSendOFPacketIn!
  \item calculating statistics summary from all controllers using
    command \lstinline!assembleStats!
\end{itemize}

\subsubsection{Setting up a custom reaction to the controller messages}

In order to change the default reaction to the controller messages to
some custom reaction (e.g. to check controller reaction to the invalid
switch replies, packet loss etc.) command \lstinline!setUserHandler!
should be used.

\emph{Current implementation also uses this command for collecting
  statistics so it is not recommended to use both of these tools
  (statistics collection and custom reaction on controller messages)
  at the same time.}

\section{Appendices}

\subsection{Running software switches in parallel}

In order to run one (or more) software switch in parallel with another
and also to control program execution time we use
async\footnote{http://hackage.haskell.org/package/async} library.

The most simple method of using this library is limiting switch
program execution time:

\begin{lstlisting}
import Control.Concurrent (threadDelay)
import Control.Concurrent.Async (race_)

...
race_ action1 (threadDelay 1000000)
\end{lstlisting}

Here we assume that action1 contains the start of some switch program so the
execution of this line will finish either on switch program
termination (if it does not use default implementation and is finite
in time) or after one second of time.

If more than one switch should be run simultaneously then typical
solution in this case will be to run each of the switches in separate
thread (using \lstinline!async!) and to wait for their termination (or
till some period of time expires).

Schematically it will look like:

\begin{lstlisting}
  a1 <- async $ withSwitch sw1 ...
  ...
  aN <- async $ withSwitch swN ...
  mapM_ waitCatch [a1,....,aN]
\end{lstlisting}

If program should be terminated if some error occurs in any of the
switches then \lstinline!waitCatch! should be replaced with
\lstinline!wait!.

Instead of \lstinline!withSwitch! command \lstinline!runSwitch! could
be used depending on the requirements.

\subsection{Program examples}

\subsubsection{Simple switch}

Here is a quite simple example in which we create 2
switches. Configuration of the second one is outputted to the console and
the first one runs the default program with time limitation of 3 seconds.

\begin{lstlisting}
{-# LANGUAGE OverloadedStrings #-}
module Main
  where

import Control.Concurrent (threadDelay)
import Control.Concurrent.Async (race_) -- EDSL for asynchronous actions
import Data.Bits                        -- for IP creation
import HCProbe.EDSL

main = do 
    (sw1,sw2) <- config $ do
      sw1 <- switch $ do
            features $ do
                {- replicateM 48 $ --uncomment to create 48 ports -}
                addPort [] [] [OFPPF_1GB_FD, OFPPF_COPPER] def
                addPort [] [] [OFPPF_1GB_FD, OFPPF_COPPER] def
            addMACs [1..450]
      sw2 <- switchOn sw1 $ do
                clearMACs 
                addMACs [400..500]
      return (sw1,sw2)
    print sw2
    race_ (runSwitch sw1 "localhost" 6633) (threadDelay 3000000)
\end{lstlisting}

\end{document}
